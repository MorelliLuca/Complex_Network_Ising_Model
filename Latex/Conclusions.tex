\section{Conclusions}
In this project we simulated the Ising model on different networks. Then, measuring the proprieties of the networks generated by neighbors aligned atoms, we studied the behavior of the system in different temperatures.

First, we compared the critical temperatures, measured using the specific heat and the magnetic susceptibility, with the theoretical ones for the 2-D square, triangular and hexagonal lattices. All these measures showed agreement between the simulations ($T_{CV}^{Sq}=1.11\times10^2,\ T_{\chi}^{Sq}=1.13\times10^2,\ T_{CV}^{Tri}=1.78\times10^2,\ T_{\chi}^{Tri}=1.77\times10^2,\ T_{CV}^{Hex}=6.1\times10^1,\ T_{\chi}^{Hex}=6.6\times10^1$) and the exact solutions ($T^{Sq}= 1.134645\times 10^2,\ T^{Tri}= 1.82048\times 10^2,\ T^{Hex}=7.5\times 10^1$). Measurements of the magnetization showed the phase transition behavior expected in all the simulations and also the entropy, the energy and the free energy behaved in the expected way.

We also measured from all the previous networks some network quantities: this was done by splitting each network into spin up and down networks, in order to obtain graphs describing how aligned atoms group together in the lattice. These measures showed the phase transition also from the network perspective highlighting the spontaneous symmetry breaking of the system: in fact the two networks behave in totally different ways before and during the phase transition, but then they become indistinguishable. We were also able to study the fragmentation of the different lattices after the phase transition: all networks, from a single giant component, broke into to big components (one of spin up and the other of spin down) with a few satellites. The bigger ones showed a bigger diameter and a smaller connectivity, signaling that at high temperatures aligned spins are more disperse. Also, the betweenness centrality showed a rise after the phase transition in accordance to our previous considerations. We also noticed that for all lattices the connectivity always dropped to approximately 1, indicating that aligned spins group together in tree-like networks.

Lastly, we studied other networks. We tried to break the square lattices by removing some atoms in order to mimic defects of the lattices. This first analysis showed that as we removed more atoms the lattices broke in more and more connected networks of aligned spins. We also observed that the time required for the thermalization of the system increased too. The second type of network analyzed is the Erdős Renyl one: our simulations showed that this network, that could be interpreted as a system with random long and short range interactions, still posses the same phase transition behavior, however at high temperatures we get more connected networks of neighbor aligned spins. We also tried to simulate lattices in which each atom can interact with the neighbors of its neighbors: in this case we observed phase transitions at higher temperature and with only one connected component for spin type at high temperatures.  Lastly, we simulated 1D lattices (in closed loops and not) and unexpectedly we still observed a phase transition.

We can conclude that the direct measure of the proprieties of the networks generated by aligned spins still shows the phase transition of the Ising model through the spontaneous symmetry breaking behavior of these systems. These measures allow also to study how aligned spins group together: whether they form a single group or not and how these groups are.

