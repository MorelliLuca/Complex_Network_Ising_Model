\section{Introduction}
Due to its simplicity, while already showing really important behaviors like spontaneous symmetry breaking, the Ising model is probably the most well known model of theoretical physics. The model describe interactions between neighbors atoms that have spin, either $+1$ or $-1$, and thus are part of a body that exhibit ferromagnetic behavior. These systems show a phase transition at a critical temperature: before this temperature the system behaves magnetically while after it stops to be magnetic. 

All the atoms are arranged on a lattice that represent the structure of the material: network theory can be used to study the dynamics of this model and how the lattices proprieties can influence the overall behavior. A network is a collection of nodes connected by links, each network is characterized by how its nodes are connected: in our model every atom is represented by a node with each interacting pair connected by a link. 

In this project we simulate numerically the Ising model on different kind of networks using the Hastings-Metropolis algorithm. This algorithm explores the phase space of the system simulating thermal fluctuations that bring the system to equilibrium at a defined temperature. At different temperatures then we measure the main thermodynamic variables and also some relevant network quantities, these allow us to detect phase transitions in the system and then to compare the measured critical temperatures with the theoretical predictions. The network quantities are also used to study how the networks behaves during a phase transitions: in particular these allow us to understand how small groups of aligned atoms form during the transition. 


